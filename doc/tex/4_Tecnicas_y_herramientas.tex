\capitulo{4}{Técnicas y herramientas}

\section{Técnicas metodológicas}
El desarrollo del proyecto seguira un enfoque basado en la metodologías agile y prácticas DevOps, con el objetivo de garantizar el desarrollo eficiente y la entrega continua de la aplicación. Seguidamente, se describiran en mas detalle las tecnicas a emplear:

\subsection{Metodologías ágiles}

La metodología ágil es un concepto de gestión de proyectos que implica dividir el proyecto en fases y hace hincapié en la colaboración y la mejora continuas. Los equipos siguen un ciclo de planificación, ejecución y evaluación. \cite{AtlassianAgile}

En este proyecto, se adoptará un enfoque ágil para gestionar el desarrollo, lo que permitirá trabajar en iteraciones cortas y realizar ajustes constantes. Para ello:

\begin{itemize}
    \item Se utilizará un tablero en Trello para gestionar las tareas y hacer seguimiento del progreso. El tablero estará dividido en tres columnas:  Planificación, En Progreso y Completado.
    \item Se trabajará en sprints cortos, revisando los avances en cada ciclo de desarrollo y ajustando el trabajo según sea necesario.
\end{itemize}

\subsection{Prácticas DevOps}

DevOps es una metodología de desarrollo de software que acelera la entrega de aplicaciones y servicios de mayor calidad combinando y automatizando el trabajo de los equipos de desarrollo de software y operaciones de TI. \cite{IBMDevOps}

Se aplicarán varias prácticas DevOps para mejorar la integración y el despliegue del sistema:

\begin{itemize}
    \item Se utilizará Git como sistema de control de versiones para gestionar el código fuente y las ramas de desarrollo.
    \item Se implementará un flujo de trabajo de integración continua con \textbf{GitHub Actions}, lo que permitirá automatizar las pruebas y despliegues del sistema.
    \item Se utilizará \textbf{Docker} para contenerizar la aplicación y garantizar la portabilidad y escalabilidad del sistema.
    \item Se desplegará la aplicación utilizando \textbf{Terraform} para gestionar el \textbf{Iac} (Infrastructure as Code) y garantizar la reproducibilidad del entorno de producción.
    \item Se implementará un sistema de monitorización y registro de errores con \textbf{Prometheus} y \textbf{Grafana} para garantizar la disponibilidad y el rendimiento del sistema.
\end{itemize}

\section{Herramientas de desarrollo}

En el proyecto se utilizarán diversas herramientas de desarrollo para facilitar la implementación y el despliegue de la aplicación. A continuación, se describen las herramientas a usar:

\subsection{Backend}

\begin{itemize}
    \item Lenguaje de programación: \textbf{Python 3.13}
    \item Framework web: \textbf{Flask} y \textbf{Django}
    \item Base de datos: \textbf{PostgreSQL}
    \item Gestor de paquetes: \textbf{Miniconda} y \textbf{Pip}
    \item Procesamiento de lenguaje natural (PLN): \textbf{Amazon Bedrock Claude Haiku 3.5} y \textbf{Comprehend Medical}
\end{itemize}

\subsection{Frontend}

\begin{itemize}
    \item Lenguaje de programación: \textbf{JavaScript}
    \item Framework de desarrollo: \textbf{React Native}
    \item Entorno de ejecución: \textbf{Expo}
    \item Gestor de paquetes: \textbf{NPM} (Node Package Manager)
    \item Biblioteca de UI: \textbf{NativeWind}
    \item Navegación: \textbf{React Navigation}
\end{itemize}

\subsection{Control de versiones}

\begin{itemize}
    \item Sistema de control de versiones: \textbf{Git}
    \item Plataforma de alojamiento de repositorios: \textbf{GitHub}
\end{itemize}

\subsection{Integración continua}

\begin{itemize}
    \item Plataforma de integración continua: \textbf{GitHub Actions}
\end{itemize}

\subsection{Contenerización}

\begin{itemize}
    \item Plataforma de contenerización: \textbf{Docker}
    \item Orquestador de contenedores: \textbf{Docker Compose}
\end{itemize}

\subsection{Despliegue e infraestructura}

\begin{itemize}
    \item Herramienta de despliegue: \textbf{Terraform}
    \item Proveedor de servicios en la nube: \textbf{AWS} (Amazon Web Services)
\end{itemize}

\subsection{Monitorización}

\begin{itemize}
    \item Herramientas de monitorización: \textbf{Prometheus} y \textbf{Grafana}
\end{itemize}

\subsection{Otras herramientas}

\begin{itemize}
    \item Gestión de tareas y metodología ágil: \textbf{Trello}
    \item Redacción y documentación: \textbf{LaTeX}
\end{itemize}
