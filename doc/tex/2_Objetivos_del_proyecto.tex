\capitulo{2}{Objetivos del proyecto}

\section{Objetivos generales}

Desarrollar un chatbot para realizar un triaje automatizado, con el objetivo de reducir los tiempos de espera y la demanda excesiva sobre el personal sanitario, con un enfoque específico en entornos laborales y educativos.

\section{Objetivos específicos}

\begin{itemize}
    \item Diseñar un chatbot donde el usuario pueda indicar sus síntomas y encontrar un diagnostico presuntivo.
    \item Implementar un sistema de analisis automatizado aplicando el Sistema español de triaje (SET) y la escala númerica de intesidad de dolor (ENV).
    \item Integrar tecnologías como Amazon Bedrock Claude Haiku 3.5 y Comprehend Medical para generar un presunto diagnostico.
    \item Desarrollar una arquitectura robusta, escalable y garantizando la seguridad y privaciada de los datos sensibles.
    \item Facilitar el acceso a un diagnóstico inicial sin la necesidad de intervención humana directa.
    \item Optimizar la experiencia del usuario, garantizando una interacción amigable y eficiente con el chatbot.
\end{itemize}