\capitulo{1}{Introducción}

En un ámbito de la salud, unos de los mayores desafíos actuales son la saturación de la atencion primaria, lo que provoca largas esperas y sobrecarga laboral de los profesionales de la salud. Con el avance de la tecnología, se han explorados diversas soluciones para agilizar el procesos del diagnostico de los pacientes. Este trabajo de fin de grado se enfoca en el desarrollo de un chatbot diseñado para realizar un triaje automatizado, donde el triaje se entiende como un proceso de clasificación que permite evaluar los riesgos clínicos, optimizar los recursos y priorizar la atención a los pacientes según la gravedad de su condición. El objetivo principal de este proyecto es reducir los tiempos de espera y la demanda excesiva sobre el personal sanitario, con un enfoque específico en entornos laborales y educativos.

El chatbot permitira a los usuarios introducir sus síntomas, a partir el sistema realizara el analisis de los datos introducidos por el usuario utilizando tecnología como Amazon Bedrock Claude Haiku 3.5 y Comprehend Medical para generar un presunto diagnostico. Además, se implementará una lógica de clasificación de triaje basada en el Sistema español de triaje (SET) y la escala númerica de intesidad de dolor (ENV). Esto permitirá clasificar los casos de manera más eficiente y precisa, garantizando que los usuarios reciban la atención adecuada según su condición.

Para el desarrollo de la aplicación, se empleará una arquitectura robusta, escalable y enfocada en la seguridad de los datos sensibles. Para ello, se utilizarán diversas herramientas en la creación del backend, como Flask y Django. Este proyecto tiene un enfoque claro en la mejora de la eficiencia en los servicios de salud, facilitando el acceso a un diagnóstico inicial sin la necesidad de intervención humana directa, lo cual es especialmente relevante en contextos con alta demanda de atención médica, como los entornos educativos y laborales.

\section{Objetivos del proyecto}

\subsection{Objetivos generales}

Desarrollar un chatbot para realizar un triaje automatizado, con el objetivo de reducir los tiempos de espera y la demanda excesiva sobre el personal sanitario, con un enfoque específico en entornos laborales y educativos.

\subsection{Objetivos específicos}

\begin{itemize}
    \item Diseñar un chatbot donde el usuario pueda indicar sus síntomas y encontrar un diagnostico presuntivo.
    \item Implementar un sistema de analisis automatizado aplicando el Sistema español de triaje (SET) y la escala númerica de intesidad de dolor (ENV).
    \item Integrar tecnologías como Amazon Bedrock Claude Haiku 3.5 y Comprehend Medical para generar un presunto diagnostico.
    \item Desarrollar una arquitectura robusta, escalable y garantizando la seguridad y privaciada de los datos sensibles.
    \item Facilitar el acceso a un diagnóstico inicial sin la necesidad de intervención humana directa.
    \item Optimizar la experiencia del usuario, garantizando una interacción amigable y eficiente con el chatbot.
\end{itemize}