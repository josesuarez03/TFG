\capitulo{1}{Introducción}

En un ámbito de la salud, unos de los mayores desafíos actuales son la saturación de la atencion primaria, lo que provoca largas esperas y sobrecarga laboral de los profesionales de la salud. Con el avance de la tecnología, se han explorados diversas soluciones para agilizar el procesos del diagnostico de los pacientes. En este trabajo de fin de grado se enfoca en el desarrollo de un chatbot diseñado para realizar un triaje automatizado, donde triaje se entiende como un proceso de clasificación que nos permite evaluar los riesgos clínicos para manejar mejor los recursos y priorizar dichos recursos para un paciente. El objetivo principal consiste en reducir los tiempos de espera y la sobrecarga de los profesionales de la salud, este proyecto estara enfocado en entornos laborales y educativos.

El chatbot permitira a los usuarios introducir sus síntomas, a partir el sistema realizara el analisis de los datos introducidos por el usuario utilizando tecnología como Amazon Bedrock Claude Haiku 3.5 y Comprehend Medical para generar un presunto diagnostico. Además, se implementará una lógica de clasificación de triaje utilizando el Sistema español de triaje (SET) y la escala de intesidad de dolor númerica (ENV). Esto ayudará a clasificar de manera eficiente y mas precisa permitiendo que los usuarios puedan recibir un atención adecuada según la condición.

