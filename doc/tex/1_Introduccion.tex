\capitulo{1}{Introducción}

En el ámbito de la salud, uno de los mayores desafíos actuales es la \textbf{saturación de la atención primaria}, lo que provoca \textbf{largos tiempos de espera} y una \textbf{sobrecarga laboral} para los profesionales sanitarios. Con el avance de la tecnología, se han explorado diversas soluciones para \textbf{agilizar el proceso de diagnóstico} de los pacientes.  

Este \textbf{Trabajo de Fin de Grado} se centra en el desarrollo de un \textbf{chatbot de triaje automatizado}, donde el triaje se define como un \textbf{proceso de clasificación} que permite \textbf{evaluar riesgos clínicos, optimizar recursos} y \textbf{priorizar la atención} en función de la gravedad del paciente. El objetivo principal de este proyecto es \textbf{reducir los tiempos de espera} y \textbf{disminuir la carga del personal sanitario}, con un enfoque específico en \textbf{entornos laborales y educativos}.  

El chatbot permitirá a los usuarios \textbf{introducir sus síntomas}, tras lo cual el sistema analizará los datos ingresados utilizando tecnologías avanzadas como \textbf{Amazon Bedrock Claude Haiku 3.5} y \textbf{Comprehend Medical} para generar un \textbf{diagnóstico presuntivo}. Además, se implementará una lógica de clasificación basada en el \textbf{Sistema Español de Triaje (SET)} y la \textbf{Escala Numérica de Intensidad del Dolor (ENV)}, lo que permitirá clasificar los casos de manera más eficiente y precisa, asegurando que los usuarios reciban la atención adecuada según su condición.  

Para el desarrollo de la aplicación, se empleará una arquitectura \textbf{robusta, escalable y centrada en la seguridad} de los datos sensibles. Se utilizarán diversas herramientas para la creación del backend, como \textbf{Flask y Django}, garantizando un sistema confiable y de alto rendimiento. Este proyecto tiene como propósito mejorar la eficiencia en los servicios de salud, facilitando el acceso a un diagnóstico inicial \textbf{sin necesidad de intervención humana directa}, lo que resulta especialmente útil en \textbf{contextos con alta demanda de atención médica}, como los entornos educativos y laborales.  

\bigskip
\noindent A continuación, se detallan los objetivos generales y específicos que se buscan alcanzar en este proyecto:

\subsection{Objetivo general}  

Desarrollar un \textbf{chatbot de triaje automatizado} con el propósito de \textbf{reducir los tiempos de espera} y \textbf{disminuir la carga laboral del personal sanitario}, enfocándose en su aplicación en \textbf{entornos laborales y educativos}.  

\subsection{Objetivos específicos}  

\begin{itemize}
    \item Diseñar un chatbot que permita a los usuarios \textbf{ingresar sus síntomas} y recibir un \textbf{diagnóstico presuntivo}.  
    \item Implementar un sistema de \textbf{análisis automatizado} basado en el \textbf{Sistema Español de Triaje (SET)} y la \textbf{Escala Numérica de Intensidad del Dolor (ENV)}.  
    \item Integrar tecnologías avanzadas como \textbf{Amazon Bedrock Claude Haiku 3.5} y \textbf{Comprehend Medical} para mejorar la precisión del diagnóstico.  
    \item Desarrollar una arquitectura \textbf{robusta y escalable}, garantizando la \textbf{seguridad y privacidad} de los datos sensibles.  
    \item Facilitar el acceso a un diagnóstico inicial \textbf{sin intervención humana directa}.  
    \item Optimizar la \textbf{experiencia del usuario}, garantizando una interacción \textbf{fluida, intuitiva y eficiente} con el chatbot.  
\end{itemize}  