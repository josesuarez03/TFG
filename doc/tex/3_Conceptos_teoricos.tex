\capitulo{2}{Conceptos teóricos}

El desarrollo de un chatbot que automatice el proceso de triaje médico se presenta como una solución viable debido al avance de tecnologías como la inteligencia artificial y el procesamiento de lenguaje natural. Estos avances, combinados con la necesidad de optimizar la atención primaria y reducir la saturación en los servicios de salud, respaldan la viabilidad de desarrollar e implementar un chatbot para el triaje médico.

Además, la integración de herramientas como Amazon Comprehend Medical permite extraer y analizar datos clínicos de manera segura y eficiente, garantizando el cumplimiento de las normativas de protección de datos. Asimismo, el uso de modelos avanzados de inteligencia artificial mejora la capacidad del chatbot para comprender y responder a las consultas de los usuarios con precisión y eficiencia, optimizando el proceso de clasificación y priorización de pacientes.

\section{Evolución del Triaje Médico}

El triaje es un proceso de clasificación que nos permite identificar y priorizar a los pacientes en función de la gravedad de su estado de salud. El objetivo principal es una asistencia eficaz y eficiente, y por tanto, una herramienta rápida, fácil de aplicar y que además poseen un fuerte valor predictivo de gravedad, de evolución y de utilización de recursos. \cite{SOLER2010}

El origen del triaje se remonta a la época napoleónica, cuando se implementó por primera vez como un mecanismo para clasificar a los heridos en el campo de batalla según la urgencia de su atención. No fue hasta 1964 cuando este sistema se adoptó en el ámbito civil, estableciendo una clasificación inicial de tres niveles: pacientes no urgentes, urgentes y emergentes. A partir de la década de 1990, el triaje evolucionó a una escala de cinco niveles, expandiéndose a diversos países con sistemas propios, como la Australia Triage Scale en Australia, el Canadian Emergency Department Triage and Acuity Scale en Canadá y el Manchester Triage System en el Reino Unido, desarrollado en la ciudad homónima.

En España, el Sistema Estructurado de Triaje (SET) se ha consolidado como el estándar en la clasificación de pacientes en los servicios de urgencias. Este será el sistema base sobre el cual operará el chatbot desarrollado en este proyecto. \cite{eswiki_triaje}

Además de la clasificación por niveles de urgencia, el triaje emplea herramientas complementarias para evaluar la gravedad de los pacientes. Entre ellas, destacan las escalas de evaluación del dolor, siendo la Escala Numérica de Intensidad del Dolor (ENV) una de las más utilizadas. En este proyecto, se implementará esta escala para mejorar la precisión en la valoración del estado del paciente, permitiendo una clasificación más efectiva en función de la intensidad del dolor percibido.
Este método consiste en que el paciente puntúe su dolor del 0 al 10, siendo 0 la ausencia de dolor y 10 el peor dolor imaginable. \cite{EscalaDolorCM}
\subsection{Sistema Estructurado de Triaje}

El \textbf{Sistema Estructurado de Triaje} es un modelo de clasificación que organiza a los pacientes en cinco categorías según la gravedad de su cuadro clínico. Su principal objetivo es priorizar la urgencia del paciente sobre cualquier otro criterio estructural o profesional, garantizando una atención eficiente dentro del modelo de especialización en urgencias. Este sistema evalúa tanto el nivel de urgencia como la gravedad del caso, considerando el motivo de consulta como un factor determinante en la clasificación. \cite{SET_triaje}\cite{triaje_espana}

\section{Aplicación de inteligencia artificial en el sector de la salud}